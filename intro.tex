
\section{Introduction}
\setcounter{page}{1}
Enhancing the performance of an aircraft has always been a vital field of research ever since the first aircraft started to fly.
Since there has been limited improvement in the design of airfoils in the last few decades; improving aircraft performance must come from another source.
This improvement could be from propulsion, new composite materials (reducing weight), flow control.
While all three of those would be greatly beneficial to improving aircraft performance this review will focus on flow control.

The purpose of flow control is to use small features/mechanisms to perturb the flow in a beneficial manner.
This perturbation could have the intent to do a 
%A way to improve system's performance involving fluid flow is through the use of flow control methods.
Flow control methods come in two major categories, passive and active flow control techniques.
Each of these methods has its advantages and disadvantages.
Along side these methods, the aerodynamic goal of the flow control will differ from application to application.
There are typically two goals
Flow control methods come in a variety of different techniques each with different applications and goals.

The idea of controlling flow around an airfoil surface has been around for over 80 years.
This all started with NACA experiements with laminar flow control back in the 1930s.

\begin{figure}[h!]
\centering
\includegraphics[scale=1.7]{universe}
\caption{The Universe}
\label{fig:universe}
\end{figure}